\newpage
\section{Ogólne określenie wymagań}	%1
\subsection{Cel projektu} %1.1
%Określenie celu pracy, co chcemy uzyskać, jakie przewidujemy wyniki
\hspace{0.60cm}Celem projektu jest zaprojektowanie i implementacja listy dwukierunkowej działającej na stercie w języku programowania C++, która dostarcza następującą funkcjonalność:

\begin{itemize}
    \item Dodawanie elementu na początek listy.
    \item Dodawanie elementu na koniec listy.
    \item Dodawanie elementu pod wskazany indeks.
    \item Usuwanie elementu z początku listy.
    \item Usuwanie elementu z końca listy.
    \item Usuwanie elementu z pod wskazanego indeksu.
    \item Wyświetlanie całej listy.
    \item Wyświetlanie listy w odwrotnej kolejności.
    \item Wyświetlanie następnego elementu.
    \item Wyświetlanie poprzedniego elementu.
    \item Czyszczenie całej listy.
\end{itemize}


\subsection{Technologia i jezyk programowania}  %1.1       

\hspace{0.60cm}Projekt zostanie zaimplementowany w języku programowania C++ przy użyciu środowiska Code::Blocks. W ramach projektu będziemy korzystać z narzędzi i technologii umożliwiających zarządzanie kodem źródłowym za pomocą systemu kontroli wersji Git.



\subsection{Prezentacja w Github}  %1.2

\hspace{0.60cm}W celu kontroli wersji i udostępnienia projektu społeczności, zostało utworzone konto na platformie GitHub. Kod źródłowy projektu był hostowany na GitHubie, co umożliwiło śledzenie zmian w kodzie oraz udostępnienie go.

 
 
 
 
 